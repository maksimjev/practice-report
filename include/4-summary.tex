\section{Rezultatai išvados ir pasiūlymai}

\subsection{Išvados ir apibendrinimai}

Praktika įmonėje UAB \textit{„INVENTI“} suteikė galimybę išbandyti savo programavimo įgūdžius vystant telekomunikacijų verslo sistemą,
kurioje naudojamos didelė aibė skirtingų technologijų, tai leido įgyti daug skirtingos patirties. Teko išmokti dirbti pagal Agile metodologiją,
atlikti darbų vertinimus, planuoti sprintus, graudžiai bendrauti su klientu, išsiaiškinti jo poreikius sistemai, teikti pasiūlymus.

Teko susipažinti su nedidelės IT įmonės kasdienybe, įgauti nuolatinio tobulinimosi ir mokymosi mentalitetą. Darbas komandoje patobulino komunikavimo įgūdžius,
diskusijos su komandos nariais leido pasirinkti geriausius ir optimaliausius sprendimus, taip pateisinant kliento lūkesčius.

\subsection{Praktikos darbo privalumai ir trūkumai}

\subsubsection{Privalumai}

\begin{itemize}
    \item Vienas svarbiausių privalumų praktikos metu, buvo universitete įgytų žinių taikymas. Trečiame kurse dalyko „Interneto technologijos“ įsisavintos žinios leido greičiau
    atlikti UI darbus su \textit{„React“} biblioteka.
    \item Didžioji dalis programų sistemų dalykų orientuoti į komandinį darbą - tai atsispindėjo praktikos atlikimo metu, įsilieti į komandinį darbą buvo pakankamai lengva,
    tačiau komandos atsakomybė žymiai didesne - komandos narių klaidos gali įtakoti kliento pasitenkinimą, o didelė klaida gali kainuoti įmonei prarastą klientą, taip paveikiant
    įmonės pajamas.
    \item Įgyta daug patirties iš labiau patyrusių kolegų, iš vidinių mokymų, naujų technologijų nagrinėjimo.
\end{itemize}

\subsubsection{Trūkumai}
Praktikoje didelių trūkumų nepastebėta. Organizaciniai klausimai buvo sprendžiami sklandžiai, įmonės profesionalumo lygis aukštas. Tačiau galėčiau išskirti šiuos dalykus:
\medskip

\begin{itemize}
    \item Išdirbtų darbo valandų pildymas. Išdirbtą laiką reikėjo fiksuoti ne tik \textit{„Clockify“} sistemoje, bet ir \textit{„Jira“}. Kadangi vartotojų istorijos ir klaidos buvo registruojamos
    kliento projektų valdymo sistemoje \textit{„Phabricator“}, iš kurios buvo perkeliamos į „Jira“, užtrukdavo susigaudyti, prie kurios užduoties ir kiek laiko reikia užpildyti.
    \item Kliento pateikiamos vartotojo istorijos ne visada atspindėjo visą vaizdą, nebuvo iki galo aišku, kokį funkcionalumą reikia įgyvendinti, todėl analizės etapas užtrukdavo ilgiau
    negu buvo planuojamas.
\end{itemize}


\subsection{Pasiūlymai}

\subsubsection{Pasiūlymai įmonei}
Kadangi didelių trūkumų nepastebėta, tačiau galimi tam tikri procesų patobulinimai. Darbo įrankių integravimas, pavyzdžiui, užduotys, sukurtos kliento „Phabricator“ sistemoje,
tam tikro įrankio dėka galėtų būti automatizuotai perkeltos į įmonės sistemą - „Jira“. Tai liečia ir darbo valandų pildymą. Kadangi „Clockify“ pildomas laikas
gali būti nesusijęs su projektu, pavyzdžiui, vidiniai mokymai, renginiai ir t.t., galima būtų „Clockify“ sistemą integruoti su „Jira“, taip pildant laiką „Clockify“,
jeigu laiko pildymo esybėje yra „Jira“ užduoties numeris - automatiškai tos užduoties laikas pildomas ir „Jira“ sistemoje.
Taip pat būtų galima automatizuoti užduočių migravimą iš kliento „Phabricator“ sistemos į įmonės „Jira“ sistemą.

\subsubsection{Pasiūlymai universitetui}
Ši praktikos vieta įmonėje \textit{„INVENTI“} man nėra pirmoji darbo vieta IT įmonėje, pirmoji buvo  \textit{„Metasite“}, kurioje pradirbau virš metų, o dirbti pradėjau
po 3 semestro. Dėl šios priežasties, studijuojant gavosi atvirkščiai, kad žinias, įgytas darbovietėje, taikiau atlikinėjant užduotis universitete.
Todėl siūlymas būtų privalomą profesinę praktiką studentams atlikti anksčiau, pavyzdžiui, 6 arba 7 semestre. Tai leistų studentams įgavus darbinės patirties
pagrįstai rinktis pasirenkamuosius dalykus, išsirinkti, kokių žinių jiems trūksta, arba ką reikėtų išmokti.

Taip pat pastebėjau, kad man trūksta teorinių žinių apie REST ir SOAP protokolus. O su jais, mano manymu, susiduria didžioji dalis programuotojų.
Iš kolegos, studijuojančio Informatiką sužinojau, kad Informatikos studentams 6 semestre dėstomas privalomas dalykas „Web servisai“,
kuris apima šių protokolų teorines žinias bei protokolų įgyvendinimą. Manau, šio dalyko įtraukimas į studijų programą ženkliai palengvintų studentų įsiliejimą į darbinę aplinką.
