\sectionnonum{Įvadas}
Profesinė praktika buvo atliekama 2011 metais įkurtoje įmonėje UAB „INVENTI“. Ši įmonė pasirinkta dėl kelių priežasčių:

\begin{enumerate}
    \item \textbf{Vykdomi projektai.} Įmonėje vykdomi skirtingi ir besikeičiantys projektai. Įmonės klientų verslo sritys - nuo telekomunikacijų iki finansinio sektoriaus.
    Dėl šios priežasties, galima įgauti ne tik programavimo žinių, bet ir įsigilinti į projekto verslo sritį. Susitarus, yra galimybė pakeisti projektą su kuriuo dirbama.
    \item \textbf{Įmonės darbo aplinka.} Įmonė rūpinasi savo darbuotojų tiek fizine, tiek psichologine sveikata, kiekvieną mėnesį organizuojamos išvykos į renginius, pavyzdžiui,
    apsilankymas virtualios realybės arenoje, išvyka į krepšinio varžybas, pokerio, stalo žaidimų turnyrai, protmušiai. Patys darbuotojai organizuoja bėgimus, dalyvauja bėgimo
    varžybose. Kadangi įmonės kolektyvas nėra didelis, šios ataskaitos rašymo metu - 33 darbutojai, yra galimybė su visais susipažinti ir pabendrauti.
    Taip pat kiekvienais metais įmonė vykdo „workation“ praktiką. „Workation“ yra tendencija, kai įmonės kolektyvas išvyksta į užsienio šalį padirbėti nuotoliniu būdu.
    Pavyzdžiui, 2019 metais „workation“ vyko Maltoje.
    \item \textbf{Naujų ir aktualių žinių įgijimas.} Įmonė stengiasi žingsniuoti koja kojon su naujausiomis technologijų tendencijomis, pavyzdžiui, naujausi projektai parašyti su
    Kotlin programavimo kalba. Taip pat įmonės darbuotojai savarankiškai organizuoja technologijų mokymosi sesijas (angl. „tech-talk“), kuriose vienas darbuotojas dalinasi su kolektyvu
    patirtimi, PĮ tendencijomis, naujovėmis. Taip pat įmonė turi biblioteką, kurioje yra virš 120 knygų, ir jų sąrašas pastoviai pildomas.
    \item \textbf{Naudojamos technologijos.} Įmonės pagrindinė programavimo kalba yra Java/Kotlin, UI („vartotojo sąsajos“) programavimo darbams naudojama „React“ biblioteka. Tai yra privalumas,
    kadangi turiu darbo patirties dirbant su šiomis technologijomis.
    \item \textbf{Lanksčios darbo sąlygos.} Esant poreikiui, įmonė suteikia galimybę dirbti iš namų.
    Darbo valandos taip pat nėra fiksuotos, svarbiausia yra dalyvauti pokalbiuose, kurie yra planuojami iš anksto. Šios sąlygos leidžia
    puikiai suderinti darbą ir asmeninį gyvenimą.
\end{enumerate}

\smallskip

Profesinės praktikos metu buvo prižiūrima ir tobulinama „Bitė Lietuva“ vidinė klientų aptarnavimo informacinė sistema. Sistemos UI aplikacija yra parašyta su JavaScript kalba,
naudojama „React“ biblioteka. Sistemos serverinė dalis parašyta su Java programavimo kalba, naudojamas „Spring“ programavimo karkasas. Serverinė dalis yra išskaidyta į
posistemes, šis išskaidymas yra mikroservisų (angl. „microservices“) architektūrinis principas. Servisai tarpusavyje apsikeičia duomenimis per
REST (angl. „Representational State Transfer“) arba SOAP (angl. „Simple Object Access Protocol“) sąsają. Taip pat tam tikri servisai įgyvendina
CQRS (angl. „Command Query Response Segregation“) architektūrinį principą, kuris atskiria duomenų įrašymo ir duomenų užklausų sluoksnius.

Programavimui naudojama „IntelliJ IDEA“ integruota programavimo aplinka, versijų kontrolės sistema - GIT. Bitės komanda sudaryta iš vieno projekto vadovo ir trijų programuotojų.
Darbas vyksta pagal Agile metodologiją, kas dvi savaites planuojami sprintai. Projekto valdymui naudojamos dvi sistemos - vidinė „Jira“, ir kliento - „Phabricator“.

\smallskip
Praktikos laikotarpiui buvo iškelti tokie uždaviniai:
\begin{enumerate}
    \item Įgyvendinti naują funkcionalumą mikroservisų ekosistemoje taikant CQRS architektūrinį principą;
    \item Įgyvendinti vartotojo sąsajos pakeitimus naudojant React biblioteką.
\end{enumerate}

\smallskip
Prisijungus prie Bitės komandos, jau buvo pradėtas vykdyti naujas projektas - „Project TV“. Projekto esmė - įgyvendinti funkcionalumą, kuris leistų „Bitė Lietuva“ darbuotojams
per vidinę klientų aptarnavimo sistemą pardavinėti televizijos paslaugas. Kadangi sistema buvo skirta pardavinėti telekomunikacijų paslaugas,
pakeitimai buvo reikalingi visose posistemėse.

\smallskip
Praktikos atlikimo eiga buvo tokia:
\begin{enumerate}
    \item Susipažinti su Bitės komanda;
    \item Pasiruošti darbo vietą, susikonfigūruoti programavimo aplinkas, prieigą prie kliento VPN ir versijavimo sistemos;
    \item Susipažinti su projekto dokumentacija, kodo rašymo standartais;
    \item Kartu su komanda planuotis sprintus ir vykdyti paskirtas užduotis.
\end{enumerate}

\smallskip
Visų pirma buvo pristatyti Bitės komandos nariai, sistema, sistemos paskirtis ir klientas. Papasakota į ką kreiptis, prireikus pagalbos. Sekantis etapas buvo
susikonfigūruoti aplinką, atsisiųsti ir įsirašyti reikiamus įrankius, susikonfigūruoti IDE, VPN, GIT. Toliau buvo pateikti susipažinimui „Project TV“ projekto reikalavimai ir
vartotojo istorijos (angl. „User story“). Toliau vyko Bitė komandos susitikimas, kurio metu buvo planuojamas dviejų savaičių sprintas, tarp programuotojų buvo paskirstytos
užduotys. Ir galiausiai reikėjo vykdyti priskirtas užduotis, iki sekančio sprinto planavimo.